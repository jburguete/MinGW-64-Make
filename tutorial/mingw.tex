\documentclass[a4paper]{article}

\usepackage{graphicx}
\usepackage{color}

\newcommand{\FIGURE}[1]
{
	\begin{figure}[!ht]
	\centering
	\includegraphics[scale=0.35]{#1}
	\end{figure}
}

\newcommand{\RED}[1] {\textcolor{red}{#1}}

\title{Setting up a 64 bits C/Fortran/Python compiler and some useful libraries
and utilities on Microsoft Windows step by step}

\author{Javier Burguete Tolosa}

\date{\today}

\begin{document}

\maketitle

\tableofcontents

\section*{}

SORRY!: I have only one spanish license of Microsoft Windows 7. Therefore, some
figures have spanish texts.

\section{Installing MSYS}

MSYS is a minimal Unix terminal emulator running over the Microsoft Windows. In
this section we install and we do a direct access on the desktop.

\begin{enumerate}

\item Download MSYS/MinGW installer (file \RED{mingw-get-setup.exe}) on \newline
\RED{http://sourceforge.net/projects/mingw/files/Instaler}

\FIGURE{msys-1.png.eps}

\clearpage

\item Click on \RED{Save file} button
\FIGURE{msys-2.png.eps}

\item Click on the downloaded \RED{mingw-get-setup.exe} installer
\FIGURE{msys-3.png.eps}

\clearpage

\item Click on the \RED{Execute} button
\FIGURE{msys-4.png.eps}

\item Click on the \RED{Install} button
\FIGURE{msys-5.png.eps}

\clearpage

\item Click on the \RED{Continue} button (WARNING! if you change the instalation
directory you must change the /etc/fstab file on the MSYS system to work
properly)
\FIGURE{msys-6.png.eps}

\item Click on the \RED{Continue} button
\FIGURE{msys-7.png.eps}

\clearpage

\item Click only on the \RED{msys-base} option (other options install 32 bits
compilers that can interfere the 64 bits compilations)
\FIGURE{msys-8.png.eps}

\item Click on the \RED{Apply Changes} menu option
\FIGURE{msys-9.png.eps}

\clearpage

\item Click on the \RED{Apply} button
\FIGURE{msys-10.png.eps}

\item Click on the \RED{Close} button
\FIGURE{msys-11.png.eps}

\clearpage

\item Create a direct access of the
\RED{C:$\backslash$MinGW$\backslash$msys$\backslash$1.0$\backslash$msys.bat}
file sending it to the desktop
\FIGURE{msys-12.png.eps}

\item Select the new direct access and clicking on the right button and click on
the \RED{Properties} menu option
\FIGURE{msys-13.png.eps}

\clearpage

\item Click on the \RED{Change icon} button
\FIGURE{msys-14.png.eps}

\item Click on the \RED{Accept} button
\FIGURE{msys-15.png.eps}

\clearpage

\item Click ont the \RED{Check} button
\FIGURE{msys-16.png.eps}

\item Select the
\RED{C:$\backslash$MinGW$\backslash$msys$\backslash$1.0$\backslash$msys.ico} (or
\RED{C:$\backslash$MinGW$\backslash$msys$\backslash$1.0$\backslash$m.ico}) file
and click on the \RED{Accept} button
\FIGURE{msys-17.png.eps}

\clearpage

\item Select the icon and click on the \RED{Accept} button
\FIGURE{msys-18.png.eps}

\item Click on the \RED{Accept} button
\FIGURE{msys-19.png.eps}

\clearpage

\item Select the direct access clicking on the right button and click on the
\RED{Rename} menu option. Then change the label (i.e. MinGW 64 bits)
\FIGURE{msys-20.png.eps}

\item Click on the direct access button and you have a MSYS (minimal Unix)
terminal
\FIGURE{msys-21.png.eps}

\end{enumerate}

\clearpage

\section{Installing 7-Zip}

7-Zip is a files compressor required to install the MinGW 64 bits compiler. In
this section the program is installed and configured to be accessible in the
MSYS terminal.

\begin{enumerate}

\item Download the 7-Zip 64 bits installer (\RED{.msi 64-bit x64}) on \newline
\RED{http://www.7-zip.org/download.html}

\FIGURE{7zip.png.eps}

\clearpage

\item Click on \RED{Save file} button
\FIGURE{7zip-2.png.eps}

\item Click on the downloaded installer (\RED{7z920-x64.msi} is the last version
at the date to make this tutorial)

\FIGURE{7zip-3.png.eps}

\clearpage

\item Click on the \RED{Execute} button
\FIGURE{7zip-4.png.eps}

\item Click on the \RED{Next} button
\FIGURE{7zip-5.png.eps}

\clearpage

\item Accept the license and click on the \RED{Next} button
\FIGURE{7zip-6.png.eps}

\item Customize the localization files and click on the \RED{Next} button
\FIGURE{7zip-7.png.eps}

\clearpage

\item Click on the \RED{Install} button
\FIGURE{7zip-8.png.eps}

\item Click on the \RED{Yes} button
\FIGURE{7zip-9.png.eps}

\clearpage

\item Click on the \RED{Finish} button
\FIGURE{7zip-10.png.eps}

\item Go to \RED{Computer} on the main menu and click on the
\RED{System properties} menu bar option
\FIGURE{7zip-11.png.eps}

\clearpage

\item Click on the \RED{Advanced system configuration} option
\FIGURE{7zip-12.png.eps}

\item Click on the \RED{Environment variables} button
\FIGURE{7zip-13.png.eps}

\clearpage

\item Select \RED{Path} on the \RED{System variables} and click on the
\RED{Edit...} button
\FIGURE{7zip-14.png.eps}

\item Append with a semicolon the path where the 7-Zip program has been
installed (in this case \RED{C:$\backslash$Program Files$\backslash$7-Zip}) and
click on the \RED{Accept} buttons
\FIGURE{7zip-15.png.eps}

\end{enumerate}

\clearpage

\section{Installing Git}

Git is a distributed revision control and source code management. In this
section we install and configure it to be accessible in the MSYS terminal.

\begin{enumerate}

\item Download the Git installer on \RED{http://git-scm.com/download/win} and
click on \RED{Save file} button
\FIGURE{git-1.png.eps}

\clearpage

\item Click on the downloaded installer (\RED{Git-1.8.5.2-preview20131230.exe}
is the last version at the date to make this tutorial)
\FIGURE{git-2.png.eps}

\item Click on the \RED{Execute} button
\FIGURE{git-3.png.eps}

\clearpage

\item Click on the \RED{Yes} button
\FIGURE{git-4.png.eps}

\item Click on the \RED{Next} button
\FIGURE{git-5.png.eps}

\clearpage

\item Click on the \RED{Next} button
\FIGURE{git-6.png.eps}

\item Select the components to install (we use the default) and click on the
\RED{Next} button
\FIGURE{git-7.png.eps}

\clearpage

\item Select the \RED{Run Git from the Windows Command Prompt} option and click
on the \RED{Next} button
\FIGURE{git-8.png.eps}

\item Select the \RED{Checkout Windows-style, commit Unix-style line endings}
option and click on the \RED{Next} button
\FIGURE{git-9.png.eps}

\end{enumerate}

\end{document}
