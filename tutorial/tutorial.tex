\documentclass[a4paper]{article}

\usepackage{graphicx}
\usepackage{color}

\newcommand{\FIG}[2]
{
	\begin{figure}[ht!]
	\centering
	\includegraphics[scale=#1]{#2}
	\end{figure}
}
\newcommand{\FIGURE}[1]{\FIG{0.35}{#1}}
\newcommand{\FIGUREB}[1]{\FIG{0.26}{#1}}

\newcommand{\RED}[1] {\textcolor{red}{#1}}

\title{Setting up a 64 bits C/Fortran/Python compiler and some useful libraries
and tools on Microsoft Windows step by step}

\author{Javier Burguete Tolosa}

\date{\today}

\begin{document}

\maketitle

\tableofcontents

\section*{}

SORRY! I have only a spanish license of Windows 7\footnote{Windows 7 is a
registered trademark of Microsoft Corporation}. Therefore, some figures have
spanish texts.

Most of patches used in this project are extracted from:\\
\RED{https://github.com/Alexpux}

\clearpage

\section{Installing MSYS}

MSYS is a minimal Unix terminal emulator running over the Microsoft Windows. In
this section we install and we do a direct access on the desktop.

\begin{enumerate}

\item Download MSYS/MinGW installer (file \RED{mingw-get-setup.exe}) on \newline
\RED{http://sourceforge.net/projects/mingw/files/Instaler}

\FIGURE{msys-1.png.eps}

\item Click on \RED{Save file} button
\FIGURE{msys-2.png.eps}

\clearpage

\item Click on the downloaded \RED{mingw-get-setup.exe} installer
\FIGURE{msys-3.png.eps}

\item Click on the \RED{Execute} button
\FIGURE{msys-4.png.eps}

\clearpage

\item Click on the \RED{Install} button
\FIGURE{msys-5.png.eps}

\item Click on the \RED{Continue} button (WARNING! if you change the instalation
directory you must change the /etc/fstab file on the MSYS system to work
properly)
\FIGURE{msys-6.png.eps}

\clearpage

\item Click on the \RED{Continue} button
\FIGURE{msys-7.png.eps}

\item Click only on the \RED{msys-base} option (other options install 32 bits
compilers that can interfere the 64 bits compilations)
\FIGURE{msys-8.png.eps}

\clearpage

\item Click on the \RED{Apply Changes} menu option
\FIGURE{msys-9.png.eps}

\item Click on the \RED{Apply} button
\FIGURE{msys-10.png.eps}

\clearpage

\item Click on the \RED{Close} button
\FIGURE{msys-11.png.eps}

\item Create a direct access of the
\RED{C:$\backslash$MinGW$\backslash$msys$\backslash$1.0$\backslash$msys.bat}
file sending it to the desktop
\FIGURE{msys-12.png.eps}

\clearpage

\item Select the new direct access and clicking on the right button and click on
the \RED{Properties} menu option
\FIGURE{msys-13.png.eps}

\item Click on the \RED{Change icon} button
\FIGURE{msys-14.png.eps}

\clearpage

\item Click on the \RED{Accept} button
\FIGURE{msys-15.png.eps}

\item Click ont the \RED{Check} button
\FIGURE{msys-16.png.eps}

\clearpage

\item Select the
\RED{C:$\backslash$MinGW$\backslash$msys$\backslash$1.0$\backslash$msys.ico} (or
\RED{C:$\backslash$MinGW$\backslash$msys$\backslash$1.0$\backslash$m.ico}) file
and click on the \RED{Accept} button
\FIGURE{msys-17.png.eps}

\item Select the icon and click on the \RED{Accept} button
\FIGURE{msys-18.png.eps}

\clearpage

\item Click on the \RED{Accept} button
\FIGURE{msys-19.png.eps}

\item Select the direct access clicking on the right button and click on the
\RED{Rename} menu option. Then change the label (i.e. MinGW 64 bits)
\FIGURE{msys-20.png.eps}

\clearpage

\item Click on the direct access button and you have a MSYS (minimal Unix)
terminal
\FIGURE{msys-21.png.eps}

\end{enumerate}

\clearpage

\section{Installing 7-Zip}

7-Zip is a files compressor required to install the MinGW 64 bits compiler. In
this section the program is installed and configured to be accessible in the
MSYS terminal.

\begin{enumerate}

\item Download the 7-Zip 64 bits installer (\RED{.msi 64-bit x64}) on \newline
\RED{http://www.7-zip.org/download.html}

\FIGURE{7zip-1.png.eps}

\clearpage

\item Click on \RED{Save file} button
\FIGURE{7zip-2.png.eps}

\item Click on the downloaded installer (\RED{7z920-x64.msi} is the last version
at the date to make this tutorial)
\FIGURE{7zip-3.png.eps}

\clearpage

\item Click on the \RED{Execute} button
\FIGURE{7zip-4.png.eps}

\item Click on the \RED{Next} button
\FIGURE{7zip-5.png.eps}

\clearpage

\item Accept the license and click on the \RED{Next} button
\FIGURE{7zip-6.png.eps}

\item Customize the localization files and click on the \RED{Next} button
\FIGURE{7zip-7.png.eps}

\clearpage

\item Click on the \RED{Install} button
\FIGURE{7zip-8.png.eps}

\item Click on the \RED{Yes} button
\FIGURE{7zip-9.png.eps}

\clearpage

\item Click on the \RED{Finish} button
\FIGURE{7zip-10.png.eps}

\item Go to \RED{Computer} on the main menu and click on the
\RED{System properties} menu bar option
\FIGURE{7zip-11.png.eps}

\clearpage

\item Click on the \RED{Advanced system configuration} option
\FIGURE{7zip-12.png.eps}

\item Click on the \RED{Environment variables} button
\FIGURE{7zip-13.png.eps}

\clearpage

\item Select \RED{Path} on the \RED{System variables} and click on the
\RED{Edit...} button
\FIGURE{7zip-14.png.eps}

\item Append with a semicolon the path where the 7-Zip program has been
installed (in this case \RED{C:$\backslash$Program Files$\backslash$7-Zip}) and
click on the \RED{Accept} buttons
\FIGURE{7zip-15.png.eps}

\end{enumerate}

\clearpage

\section{Installing Git}

Git is a distributed revision control and source code management. In this
section we install and configure it to be accessible in the MSYS terminal.

\begin{enumerate}

\item Download the Git installer on \RED{http://git-scm.com/download/win} and
click on \RED{Save file} button
\FIGURE{git-1.png.eps}

\clearpage

\item Click on the downloaded installer (\RED{Git-1.8.5.2-preview20131230.exe}
is the last version at the date to make this tutorial)
\FIGURE{git-2.png.eps}

\item Click on the \RED{Execute} button
\FIGURE{git-3.png.eps}

\clearpage

\item Click on the \RED{Yes} button
\FIGURE{git-4.png.eps}

\item Click on the \RED{Next} button
\FIGURE{git-5.png.eps}

\clearpage

\item Click on the \RED{Next} button
\FIGURE{git-6.png.eps}

\item Select the components to install (we use the default) and click on the
\RED{Next} button
\FIGURE{git-7.png.eps}

\clearpage

\item Select the \RED{Run Git from the Windows Command Prompt} option and click
on the \RED{Next} button
\FIGURE{git-8.png.eps}

\item Select the \RED{Checkout Windows-style, commit Unix-style line endings}
option and click on the \RED{Next} button
\FIGURE{git-9.png.eps}

\clearpage

\end{enumerate}

\section{Installing Microsoft MPI (optional)}

Microsoft MPI is an implementation of the Message Passing Interface standard for
developing and running parallel applications. This step is not strictly
necessary. To install it:

\begin{enumerate}

\item Download it clicking in the \RED{Download} button from\\
\RED{http://www.microsoft.com/en-sg/download/details.aspx?id=39961}
\FIGURE{mpi-1.png.eps}

\clearpage

\item Select the correct option (\RED{msmpi\_x86.msi} for 32 bits and
\RED{msmpi\_x64.msi} for 64 bits) and click on the \RED{Next} button
\FIGURE{mpi-2.png.eps}

\item Click on the \RED{Save file} button
\FIGURE{mpi-3.png.eps}

\clearpage

\item Click on the downloaded installer (\RED{msmpi-x64.msi})
\FIGURE{mpi-4.png.eps}

\item Click on the \RED{Execute} button
\FIGURE{mpi-5.png.eps}

\clearpage

\item Click on the \RED{Next} button
\FIGURE{mpi-6.png.eps}

\item Accept the license and click on the \RED{Next} button
\FIGURE{mpi-7.png.eps}

\clearpage

\item Click on the \RED{Next} button (WARNING! if you change the instalation
directory you will must change the makefile to work properly)
\FIGURE{mpi-8.png.eps}

\item Click on the \RED{Install} button
\FIGURE{mpi-9.png.eps}

\clearpage

\item Click on the \RED{Yes} button
\FIGURE{mpi-10.png.eps}

\item Click on the \RED{Finish} button
\FIGURE{mpi-11.png.eps}

\clearpage

\end{enumerate}

\section{Installing MinGW 64 bits compiler and some useful libraries and tools}

Here we install the following MSYS useful tools:
\begin{itemize}
\item bash 3.1.17
\item bison 2.4.2
\item flex 2.5.35
\item gawk 3.1.7
\item grep 2.5.4
\item groff 1.20.1
\item gzip 1.3.12
\item m4 1.4.16
\item make 3.81
\item man 1.6f
\item openssh 5.4p1
\item perl 5.8.8
\item patch 2.6.1
\item sed 4.2.1
\item tar 1.23
\item texinfo 4.13a
\item unzip 6.0
\item vim 7.3
\item wget 1.12
\item xz 5.0.3
\item zip 3.0
\end{itemize}

with the MinGW 64 bits compilers and the following useful libraries:
\begin{itemize}
\item atk 2.10.0
\item autoconf 2.69
\item automake 1.4.1
\item cairo 1.12.16
\item expat 2.1.0
\item fontconfig 2.11.0
\item freeglut 2.8.1
\item freetype 2.5.2
\item gcc 4.8.2 (C, C++, Objective C, and FORTRAN compilers)
\item gdk\_pixbuf 2.30.2
\item gettext 0.18.3.2
\item glib 2.38.2
\item gsl 1.16
\item gtk+ 2.24.22 and 3.10.6
\item libffi 3.0.13
\item libiconv 1.14
\item libpng 1.6.8
\item libtool 2.4.2
\item libxml 2.9.1
\item pango 1.36.2
\item pixman 0.32.4
\item pkg-config 0.28
\item python 2.7.6
\item readline 6.2
\item sqlite 3.8.3
\item termcap 1.3.1
\item zlib 1.2.8
\end{itemize}

\begin{enumerate}

\clearpage

\item Open the MSYS terminal clicking on the created direct access
\FIGURE{mingw-1.png.eps}

\item Download the MinGW-64-Make set of makefiles and patches by typing\\
\RED{git clone git@github.com:jburguete/MinGW-64-Make.git}
\FIGURE{mingw-2.png.eps}

\clearpage

\item Enter on the new directory and execute the \RED{make} command passing the
architecture (\RED{ARCH="32"} or \RED{ARCH="64"} for 32 bits or 64 bits
respectively) and the version (\RED{VER="stable"}, \RED{VER="testing"} or
\newline\RED{VER="experimental"} for stable, testing or experimental versions
respectively) by typing for instance:\\
\RED{cd MinGW-64-Make}\\
\RED{make ARCH="64" VER="stable"}
\FIGURE{mingw-3.png.eps}

\item Press \RED{A} key on the question
\FIGURE{mingw-4.png.eps}

\end{enumerate}

\clearpage

\section{Installing MiKTeX LaTeX (optional)}

MiKTeX is a distribution of the LaTeX text processor for Windows. We install
here the basic 64 bits distribution.

\begin{enumerate}

\item Download the distribution from \RED{http://miktex.org/download} clicking
on the \RED{Other Downloads} panel and the
\RED{Basic MiKTeX 2.9.4813 64-bit Installer} (the number is the version at the
date to make this tutorial)
\FIGUREB{Latex-1.png.eps}

\clearpage

\item Click on the \RED{Save file} button
\FIGUREB{Latex-2.png.eps}

\item Click on the downloaded installer (\RED{basic-miktex-2.9.4813-x64.exe}
is the last version at the date to make this tutorial)
\FIGUREB{Latex-3.png.eps}

\clearpage

\item Click on the \RED{Yes} button
\FIGUREB{Latex-4.png.eps}

\item Accept the license and click on the \RED{Next} button
\FIGUREB{Latex-5.png.eps}

\clearpage

\item Select the \RED{Anyone who uses the computer (all users)} option and click
on the \RED{Next} button
\FIGUREB{Latex-6.png.eps}

\item Select the installation directory and click on the \RED{Next} button
\FIGUREB{Latex-7.png.eps}

\clearpage

\item Select the prefered options and click on the \RED{Next} button
\FIGUREB{Latex-8.png.eps}

\item Click on the \RED{Start} button
\FIGUREB{Latex-9.png.eps}

\clearpage

\item Click on the \RED{Next} button
\FIGUREB{Latex-10.png.eps}

\item Click on the \RED{Close} button
\FIGUREB{Latex-11.png.eps}

\end{enumerate}

\clearpage

\section{Installing Doxygen (optional)}

Doxygen is a

\begin{enumerate}

\item Download the program from
\RED{http://www.stack.nl/\~dimitri/doxygen/download.html} clicking
on the \RED{doxygen-1.8.6-setup.exe http} option (the number is the version at
the date to make this tutorial)
\FIGURE{doxygen-1.png.eps}

\clearpage

\item Click on the \RED{Save file} button
\FIGURE{doxygen-2.png.eps}

\item Click on the downloaded installer (\RED{doxygen-1.8.6-setup.exe} is the
last version at the date to make this tutorial)
\FIGURE{doxygen-3.png.eps}

\clearpage

\item Click on the \RED{Execute} button
\FIGURE{doxygen-4.png.eps}

\item Click on the \RED{Yes} button
\FIGURE{doxygen-5.png.eps}

\clearpage

\item Click on the \RED{Next} button
\FIGURE{doxygen-6.png.eps}

\item Accept the license and click on the \RED{Next} button
\FIGURE{doxygen-7.png.eps}

\clearpage

\item Select the installation directory and click on the \RED{Next} button
\FIGURE{doxygen-8.png.eps}

\item Select the componentes and click on the \RED{Next} button
\FIGURE{doxygen-9.png.eps}

\clearpage

\item Select the start menu folder and click on the \RED{Next} button
\FIGURE{doxygen-10.png.eps}

\item Click on the \RED{Install} button
\FIGURE{doxygen-11.png.eps}

\end{enumerate}

\clearpage

\section{Installing WinMerge (optional)}

WinMerge is a tool to show and merge differences between files.

\begin{enumerate}

\item Download the program from \RED{http://winmerge.org/downloads} clicking on
the \RED{Download Now!} button
\FIGUREB{Winmerge-1.png.eps}

\clearpage

\item Click on the \RED{Save file} button
\FIGUREB{Winmerge-2.png.eps}

\item Click on the downloaded installer (\RED{WinMerge-2.14.0-Setup.exe} is the
last version at the date to make this tutorial)
\FIGUREB{Winmerge-3.png.eps}

\clearpage

\item Click on the \RED{Execute} button
\FIGUREB{Winmerge-4.png.eps}

\item Click on the \RED{Yes} button
\FIGUREB{Winmerge-5.png.eps}

\clearpage

\item Click on the \RED{Next} button
\FIGUREB{Winmerge-6.png.eps}

\item Click on the \RED{Next} button
\FIGUREB{Winmerge-7.png.eps}

\clearpage

\item Select the installation directory and click on the \RED{Next} button
\FIGUREB{Winmerge-8.png.eps}

\item Select the componentes and click on the \RED{Next} button
\FIGUREB{Winmerge-9.png.eps}

\clearpage

\item Select the start menu folder and click on the \RED{Next} button
\FIGUREB{Winmerge-10.png.eps}

\item Select the \RED{Add WinMerge folder to your system path} option and click
on the \RED{Next} button
\FIGUREB{Winmerge-11.png.eps}

\clearpage

\item Click on the \RED{Install} button
\FIGUREB{Winmerge-12.png.eps}

\item Click on the \RED{Next} button
\FIGUREB{Winmerge-13.png.eps}

\clearpage

\item Click on the \RED{Finish} button
\FIGUREB{Winmerge-14.png.eps}

\end{enumerate}

\clearpage

\section{Installing GNUPlot (optional)}

GNUPlot is a tool to draw mathematical graphicals.

\begin{enumerate}

\item Download it from
\RED{http://sourceforge.net/projects/gnuplot/files/gnuplot/4.6.4} clicking on
the \RED{Download gp463-win32-setup.exe} option (at the moment to make this
tutorial 4.6.3 was the last program version)
\FIGUREB{GNUPlot-1.png.eps}

\clearpage

\item Click on the \RED{Save file} button
\FIGUREB{GNUPlot-2.png.eps}

\item Click on the downloaded installer (\RED{WinMerge-2.14.0-Setup.exe} is the
last version at the date to make this tutorial)
\FIGUREB{GNUPlot-3.png.eps}

\clearpage

\item Click on the \RED{Execute} button
\FIGUREB{GNUPlot-4.png.eps}

\item Click on the \RED{Yes} button
\FIGUREB{GNUPlot-5.png.eps}

\clearpage

\item Select the language and click on the \RED{Yes} button
\FIGUREB{GNUPlot-6.png.eps}

\item Click on the \RED{Next} button
\FIGUREB{GNUPlot-7.png.eps}

\clearpage

\item Accept the license and click on the \RED{Next} button
\FIGUREB{GNUPlot-8.png.eps}

\item Click on the \RED{Next} button
\FIGUREB{GNUPlot-9.png.eps}

\clearpage

\item Select the installation directory and click on the \RED{Next} button
\FIGUREB{GNUPlot-10.png.eps}

\item Select the componentes and click on the \RED{Next} button
\FIGUREB{GNUPlot-11.png.eps}

\clearpage

\item Select the start menu folder and click on the \RED{Next} button
\FIGUREB{GNUPlot-12.png.eps}

\item Select the
\RED{Add application directroy to your PATH environment variable} and click on
the \RED{Next} button
\FIGUREB{GNUPlot-13.png.eps}

\clearpage

\item Click on the \RED{Install} button
\FIGUREB{GNUPlot-14.png.eps}

\item Click on the \RED{Next} button
\FIGUREB{GNUPlot-15.png.eps}

\clearpage

\item Click on the \RED{Finish} button
\FIGUREB{GNUPlot-16.png.eps}

\end{enumerate}

\clearpage

\section{Installing Subversion (optional)}

\FIGUREB{subversion-1.png.eps}
\FIGUREB{subversion-2.png.eps}
\FIGUREB{subversion-3.png.eps}
\FIGUREB{subversion-4.png.eps}
\FIGUREB{subversion-5.png.eps}
\FIGUREB{subversion-6.png.eps}
\FIGUREB{subversion-7.png.eps}
\FIGUREB{subversion-8.png.eps}
\FIGUREB{subversion-9.png.eps}
\FIGUREB{subversion-10.png.eps}
\FIGUREB{subversion-11.png.eps}

\clearpage

\section{Installing Eclipse (optional)}

\end{document}
